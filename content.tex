\section{Introdução}

Ordenar objetos, de uma maneira geral, é bastante útil pois ajuda a
organizar, buscar de forma rápida, verificar existência, entre outros
aspectos. Em computação isso não é diferente. A ordenação é utilizada em
várias áreas, por exemplo Bancos de Dados. Com isso precisamos sempre de
algoritmos de ordenação que façam o trabalho em menos tempo, consumindo
menos recursos.

\section{Métodos de Ordenação}

Para ordenar dados computacionais precisamos de métodos, que consistem
em procedimentos que dado um conjunto geram uma saída com o próximo
elemento sempre menor ou igual ao anterior, ou vice versa.

Veremos a implementação e estatísticas dos seguintes métodos de
ordenação:

\begin{itemize}
\item
  Bubble Sort
\item
  Selection Sort
\item
  Insertion Sort
\item
  Quick Sort
\end{itemize}
\subsection{Bubble Sort}

Método de ordenação bastante simples de implementar e entender. A ideia
é percorrer um vetor de dados, normalmente números, e verificar de dois
em dois qual o maior. Em caso positivo trocar um com o outro.

\subsubsection{Complexidade}

Para cada elemento é preciso percorrer o vetor, mesmo que o mesmo já
esteja ordenado. Por isso dizemos que o algoritmo é $O(n^2)$.

\subsubsection{Pseudo-código}

Esse método é caracterizado pelo uso de variável auxiliar para fazer o
\emph{swap} ou troca.

\begin{verbatim}
void bubble_sort(V[], n)
begin
    k = n - 1
    for i = 1 to n do
        j = 1
        while j <= k do
            if V[j] > V[j + 1] do
                aux = V[j]
                V[j] = V[j + 1]
                V[j + 1] = aux
            endif
            j = j + 1
        endwhile
    endfor
end
\end{verbatim}

