\documentclass[a4paper,12pt]{article}
\usepackage[utf8]{inputenc}
\usepackage[brazil]{babel}
\usepackage{graphicx, xcolor}
\usepackage{ctable}
\usepackage{amsmath}

% -- hyperlink text
\usepackage[colorlinks,
  urlcolor=blue,
  hyperindex,
  pdfdisplaydoctitle,
  pageanchor,
  linkcolor=blue]{hyperref}
% -- hyperlink text
\usepackage{url}

% -- glossary
\usepackage[toc=true,style=list]{glossary}
% ...
\makeglossary
% -- glossary

\title{Algoritmos de Ordenação}
\author{Átila Camurça}

\begin{document}

\maketitle

\begin{abstract}
Este artigo pretende mostrar métodos de ordenação e seus respectivos
resultados em relação ao número de itens ordenados, tempo de ordenação,
consumo de recursos, e ao final compará-los determinando qual o melhor em cada
situação.

\smallskip
\noindent \textbf{Palavras-chave:} ordenação, análise assintótica.

\end{abstract}

\newpage

\tableofcontents

\section{Introdução}

Ordenar objetos, de uma maneira geral, é bastante útil pois ajuda a
organizar, buscar de forma rápida, verificar existência, entre outros
aspectos. Em computação isso não é diferente. A ordenação é utilizada em
várias áreas, por exemplo Bancos de Dados. Com isso precisamos sempre de
algoritmos de ordenação que façam o trabalho em menos tempo, consumindo
menos recursos.

\section{Métodos de Ordenação}

Para ordenar dados computacionais precisamos de métodos, que consistem
em procedimentos que dado um conjunto geram uma saída com o próximo
elemento sempre menor ou igual ao anterior, ou vice versa.

Veremos a implementação e estatísticas dos seguintes métodos de
ordenação:

\begin{itemize}
\item
  Bubble Sort
\item
  Selection Sort
\item
  Insertion Sort
\item
  Quick Sort
\end{itemize}
\subsection{Bubble Sort}

Método de ordenação bastante simples de implementar e entender. A ideia
é percorrer um vetor de dados, normalmente números, e verificar de dois
em dois qual o maior. Em caso positivo trocar um com o outro.

\subsubsection{Complexidade}

Para cada elemento é preciso percorrer o vetor, mesmo que o mesmo já
esteja ordenado. Por isso dizemos que o algoritmo é $O(n^2)$.

\subsubsection{Pseudo-código}

Esse método é caracterizado pelo uso de variável auxiliar para fazer o
\emph{swap} ou troca.

\begin{verbatim}
void bubble_sort(V[], n)
begin
    k = n - 1
    for i = 1 to n do
        j = 1
        while j <= k do
            if V[j] > V[j + 1] do
                aux = V[j]
                V[j] = V[j + 1]
                V[j + 1] = aux
            endif
            j = j + 1
        endwhile
    endfor
end
\end{verbatim}



\appendix

\section{Sobre o Autor}

Estudante de Ciência da Computação no IFCE Campus Maracanaú, Ceará, Brazil. Programador Java,
PHP, Ruby. Trabalha na Fidias Software, empresa que ajudou a fundar juntamente com José Alberto
e Shara Shami.

\subsection*{Sugestões e Críticas}

Para fazer sugestões e críticas envie e-mail para \texttt{camurca.home@gmail.com}. Para ficar antenado
no mundo do Software Livre me acompanhe no Twitter:\\ \url{https://twitter.com/#!/atilacamurca}.

\subsection*{Código-fonte deste artigo}

Você pode baixar o código-fonte deste artigo em\\ \url{https://github.com/atilacamurca/article-sorting-algorithms}.

\subsection*{Código-fonte do projeto}

O código-fonte da implementação dos algoritmos pode ser baixado em\\ \url{https://github.com/atilacamurca/sorty}

\nocite{harris}
\nocite{santos}
\nocite{hshah}
\nocite{lydia}
\nocite{rosetta-counting}
\nocite{fong}
\nocite{robson}
\nocite{algolist-selection}
\nocite{unicamp-insertion}
\nocite{rmuhamma}

\bibliographystyle{ieeetr}
\bibliography{reference}% expects file "reference.bib"

\end{document}
